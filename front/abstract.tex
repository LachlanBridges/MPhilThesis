%!TEX root = ../thesis.tex
\begin{chapter}{Abstract}
\label{ch:abstract}
Early theoretical models of animal foraging determined that L\'{e}vy flights were an optimal search strategy in a number of different scenarios.
However, a new family of strategies known as \emph{intermittent} or \emph{regime-switching} strategies have been found to provide a higher search efficiency.
In this thesis, we investigate regime-switching strategies using Markov-modulated random walks. Our model allows a forager to have any number of different search strategies that it switches between according to some Markov chain.
We derive an expression for the efficiency of a Markov-modulated random walk, and develop discrete approximations in order to solve our model numerically.
We are able to show that many of the existing strategies investigated throughout the literature, such as giving-up time strategies, can be seen as a special case of a Markov-modulated random walk strategy.
We are also able to approximate a search model with hidden-targets, where a forager can only locate targets while in a certain state.
Using our new expression for the efficiency we recover some existing results from the literature as well as find some new results for the optimal search strategy under various circumstances.
Finally, we outline a very simple two-dimensional model, in which food patches are distributed according to a homogeneous spatial Poisson process.
We make some simplifying assumptions about the chance of finding food when backtracking, and find an upper bound on the efficiency of a search.
We show that taking into account some backtracking makes the model too difficult to solve, and use simulations to investigate the accuracy of our model.
\end{chapter}


